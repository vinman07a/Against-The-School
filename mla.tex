\documentclass[12pt,letterpaper]{article}
\usepackage[utf8]{inputenc}
\usepackage[american]{babel}
\usepackage{csquotes}
\usepackage{ifpdf}
\usepackage{mla}
\usepackage{datetime2}

\usepackage[style=mla, backend=biber]{biblatex}
\defbibheading{bibliography}{\newpage\centering Works Cited}
\addbibresource{main.bib}
\setlength{\bibhang}{0.5in}

% Using \mycite[]{} over \autocite[]{} will prevent biblatex-mla from ever
% shortening your citation. My teachers have all required this.
\newcommand{\mycite}[2][]{\mancite\autocite[#1]{#2}}

\DTMnewdatestyle{MLAdate}{%
  \renewcommand*{\DTMdisplaydate}[4]{\DTMenglishordinal{##3} %
    \DTMenglishmonthname{##2} ##1}%
}
\DTMsetdatestyle{MLAdate}

\begin{document}
\begin{mla}{Vincent}{Insinga}{Freya Fitzpatrick}{High School Writing}%
  {\today}{Against The School: An Argument Against Authoritarian Education}     

  A primary issue of a large, political, and compulsorary system of
  education is the amount of alienation which it causes. Students are alienated
  from teachers by means of the machinery which controls the functioning of the
  entire system. Those same teachers are alienated from the curriculum which
  they are teaching by faculty, buereacracy, and politics. In this way, each
  aspect of the educational system is alienated from each other
  aspect until what is left is a mechanical and industrious system which
  destroys the autonomy of all parties involved in the system. In such an
  alienating system, the only forces which can act autonomously are money money
  in the form of political movement. It seems that alienation is inseperable
  from large state operations, for this reason I argue that education must be
  done at a small scale at a local level; unfortunately due to the beuracratic
  nature of all state programs it is clear that this can not be done at a state
  or federal level \autocite{atl}.

  Part of the way that this alienation developed was through the
  commodification of education. Education can not be treated as a public good
  becuase it can not be treated as a commodity. An often touted response to
  public education is one base on a free market in which parents are customers
  and schools are suppliers; this is an innefective strategy because it ignores
  the root issue of alienation and will ultimately result, due to the
  concentrative nature of these things, in a new highly centralized system of
  beuracracy

\printbibliography
\end{mla}
\end{document}
